% Template for PLoS
% Version 3.5 March 2018
%
% % % % % % % % % % % % % % % % % % % % % %
%
% -- IMPORTANT NOTE
%
% This template contains comments intended
% to minimize problems and delays during our production
% process. Please follow the template instructions
% whenever possible.
%
% % % % % % % % % % % % % % % % % % % % % % %
%
% Once your paper is accepted for publication,
% PLEASE REMOVE ALL TRACKED CHANGES in this file
% and leave only the final text of your manuscript.
% PLOS recommends the use of latexdiff to track changes during review, as this will help to maintain a clean tex file.
% Visit https://www.ctan.org/pkg/latexdiff?lang=en for info or contact us at latex@plos.org.
%
%
% There are no restrictions on package use within the LaTeX files except that
% no packages listed in the template may be deleted.
%
% Please do not include colors or graphics in the text.
%
% The manuscript LaTeX source should be contained within a single file (do not use \input, \externaldocument, or similar commands).
%
% % % % % % % % % % % % % % % % % % % % % % %
%
% -- FIGURES AND TABLES
%
% Please include tables/figure captions directly after the paragraph where they are first cited in the text.
%
% DO NOT INCLUDE GRAPHICS IN YOUR MANUSCRIPT
% - Figures should be uploaded separately from your manuscript file.
% - Figures generated using LaTeX should be extracted and removed from the PDF before submission.
% - Figures containing multiple panels/subfigures must be combined into one image file before submission.
% For figure citations, please use "Fig" instead of "Figure".
% See http://journals.plos.org/plosone/s/figures for PLOS figure guidelines.
%
% Tables should be cell-based and may not contain:
% - spacing/line breaks within cells to alter layout or alignment
% - do not nest tabular environments (no tabular environments within tabular environments)
% - no graphics or colored text (cell background color/shading OK)
% See http://journals.plos.org/plosone/s/tables for table guidelines.
%
% For tables that exceed the width of the text column, use the adjustwidth environment as illustrated in the example table in text below.
%
% % % % % % % % % % % % % % % % % % % % % % % %
%
% -- EQUATIONS, MATH SYMBOLS, SUBSCRIPTS, AND SUPERSCRIPTS
%
% IMPORTANT
% Below are a few tips to help format your equations and other special characters according to our specifications. For more tips to help reduce the possibility of formatting errors during conversion, please see our LaTeX guidelines at http://journals.plos.org/plosone/s/latex
%
% For inline equations, please be sure to include all portions of an equation in the math environment.
%
% Do not include text that is not math in the math environment.
%
% Please add line breaks to long display equations when possible in order to fit size of the column.
%
% For inline equations, please do not include punctuation (commas, etc) within the math environment unless this is part of the equation.
%
% When adding superscript or subscripts outside of brackets/braces, please group using {}.
%
% Do not use \cal for caligraphic font.  Instead, use \mathcal{}
%
% % % % % % % % % % % % % % % % % % % % % % % %
%
% Please contact latex@plos.org with any questions.
%
% % % % % % % % % % % % % % % % % % % % % % % %

\documentclass[10pt,letterpaper]{article}
\usepackage[top=0.85in,left=2.75in,footskip=0.75in]{geometry}

% amsmath and amssymb packages, useful for mathematical formulas and symbols
\usepackage{amsmath,amssymb}

% Use adjustwidth environment to exceed column width (see example table in text)
\usepackage{changepage}

% Use Unicode characters when possible
\usepackage[utf8x]{inputenc}

% textcomp package and marvosym package for additional characters
\usepackage{textcomp,marvosym}

% cite package, to clean up citations in the main text. Do not remove.
% \usepackage{cite}

% Use nameref to cite supporting information files (see Supporting Information section for more info)
\usepackage{nameref,hyperref}

% line numbers
\usepackage[right]{lineno}

% ligatures disabled
\usepackage{microtype}
\DisableLigatures[f]{encoding = *, family = * }

% color can be used to apply background shading to table cells only
\usepackage[table]{xcolor}

% array package and thick rules for tables
\usepackage{array}

% create "+" rule type for thick vertical lines
\newcolumntype{+}{!{\vrule width 2pt}}

% create \thickcline for thick horizontal lines of variable length
\newlength\savedwidth
\newcommand\thickcline[1]{%
  \noalign{\global\savedwidth\arrayrulewidth\global\arrayrulewidth 2pt}%
  \cline{#1}%
  \noalign{\vskip\arrayrulewidth}%
  \noalign{\global\arrayrulewidth\savedwidth}%
}

% \thickhline command for thick horizontal lines that span the table
\newcommand\thickhline{\noalign{\global\savedwidth\arrayrulewidth\global\arrayrulewidth 2pt}%
\hline
\noalign{\global\arrayrulewidth\savedwidth}}


% Remove comment for double spacing
%\usepackage{setspace}
%\doublespacing

% Text layout
\raggedright
\setlength{\parindent}{0.5cm}
\textwidth 5.25in
\textheight 8.75in

% Bold the 'Figure #' in the caption and separate it from the title/caption with a period
% Captions will be left justified
\usepackage[aboveskip=1pt,labelfont=bf,labelsep=period,justification=raggedright,singlelinecheck=off]{caption}
\renewcommand{\figurename}{Fig}

% Use the PLoS provided BiBTeX style
% \bibliographystyle{plos2015}

% Remove brackets from numbering in List of References
\makeatletter
\renewcommand{\@biblabel}[1]{\quad#1.}
\makeatother



% Header and Footer with logo
\usepackage{lastpage,fancyhdr,graphicx}
\usepackage{epstopdf}
%\pagestyle{myheadings}
\pagestyle{fancy}
\fancyhf{}
%\setlength{\headheight}{27.023pt}
%\lhead{\includegraphics[width=2.0in]{PLOS-submission.eps}}
\rfoot{\thepage/\pageref{LastPage}}
\renewcommand{\headrulewidth}{0pt}
\renewcommand{\footrule}{\hrule height 2pt \vspace{2mm}}
\fancyheadoffset[L]{2.25in}
\fancyfootoffset[L]{2.25in}
\lfoot{\today}

%% Include all macros below

\newcommand{\lorem}{{\bf LOREM}}
\newcommand{\ipsum}{{\bf IPSUM}}


% Pandoc citation processing

\usepackage{booktabs}
\usepackage{longtable}
\usepackage{array}
\usepackage{multirow}
\usepackage{wrapfig}
\usepackage{float}
\usepackage{colortbl}
\usepackage{pdflscape}
\usepackage{tabu}
\usepackage{threeparttable}
\usepackage{threeparttablex}
\usepackage[normalem]{ulem}
\usepackage{makecell}
\usepackage{xcolor}



\usepackage{forarray}
\usepackage{xstring}
\newcommand{\getIndex}[2]{
  \ForEach{,}{\IfEq{#1}{\thislevelitem}{\number\thislevelcount\ExitForEach}{}}{#2}
}

\setcounter{secnumdepth}{0}

\newcommand{\getAff}[1]{
  \getIndex{#1}{Some Institute of Technology,Another University}
}

\providecommand{\tightlist}{%
  \setlength{\itemsep}{0pt}\setlength{\parskip}{0pt}}

\begin{document}
\vspace*{0.2in}

% Title must be 250 characters or less.
\begin{flushleft}
{\Large
\textbf\newline{Short-term real-time prediction of total number of reported COVID-19
cases in South Africa - A Bayesian Temporal Modeling Approch} % Please use "sentence case" for title and headings (capitalize only the first word in a title (or heading), the first word in a subtitle (or subheading), and any proper nouns).
}
\newline
% Insert author names, affiliations and corresponding author email (do not include titles, positions, or degrees).
\\
Belay Birlie Yimer\textsuperscript{\getAff{University of Manchester, Uk;}, \getAff{Hasselt University, Belgium.}}\textsuperscript{*},
Ziv Shkedy\textsuperscript{\getAff{Hasselt University}}\\
\bigskip
\textbf{\getAff{Some Institute of Technology}}Department 1, Street, City, State, Zip\\
\textbf{\getAff{Another University}}Department 2, Street, City, State, Zip\\
\bigskip
* Corresponding author: belaybirlie.yimer@manchester.ac.uk\\
\end{flushleft}
% Please keep the abstract below 300 words
\section*{Abstract}
To be updated.

% Please keep the Author Summary between 150 and 200 words
% Use first person. PLOS ONE authors please skip this step.
% Author Summary not valid for PLOS ONE submissions.
\section*{Author summary}
To be updated.

\linenumbers

% Use "Eq" instead of "Equation" for equation citations.
\emph{Text based on plos sample manuscript, see
\url{https://journals.plos.org/ploscompbiol/s/latex}}

\hypertarget{introduction}{%
\section{Introduction}\label{introduction}}

To be updated Here are two sample references: {[}1,2{]}.

\hypertarget{methods}{%
\section{Methods}\label{methods}}

\hypertarget{data}{%
\subsection{Data}\label{data}}

A daily record of newly diagnosed COVID-19 cases were extracted for the
period 5 March 2020 to 15 Feburary 2021, at national and provincial
level from a publicly available data repository {[}ref{]}. Data from the
first 335 days were used to fit the models.

\hypertarget{statistical-analysis}{%
\subsection{Statistical analysis}\label{statistical-analysis}}

\hypertarget{results}{%
\section{Results}\label{results}}

\begin{verbatim}
## [1] 11  3
\end{verbatim}

\begin{figure}
\includegraphics[width=0.9\linewidth]{COVIDincidenceSA_files/figure-latex/unnamed-chunk-4-1} \caption{Observed data - cummulative cases, daily cases, and cummulative tests.}\label{fig:unnamed-chunk-4}
\end{figure}

\begin{figure}
\includegraphics[width=0.9\linewidth]{COVIDincidenceSA_files/figure-latex/unnamed-chunk-6-1} \caption{Fitted and observed data}\label{fig:unnamed-chunk-6-1}
\end{figure}
\begin{figure}
\includegraphics[width=0.9\linewidth]{COVIDincidenceSA_files/figure-latex/unnamed-chunk-6-2} \caption{Fitted and observed data}\label{fig:unnamed-chunk-6-2}
\end{figure}

\begin{figure}
\includegraphics[width=0.9\linewidth]{COVIDincidenceSA_files/figure-latex/unnamed-chunk-8-1} \caption{Fitted and observed data}\label{fig:unnamed-chunk-8}
\end{figure}

\begin{table}

\caption{\label{tab:unnamed-chunk-9}Short-term predictions of total number of reported cases at the national level under the RW1 model. Estimation period 05/03/2020-08/02/2021}
\centering
\begin{tabular}[t]{l|l|r|r|l}
\hline
  & Date & Total & Prediction & Prediction Interval\\
\hline
329 & 2021-02-11 & 1484900 & 1485426 & (1172950.47-1870542.61)\\
\hline
330 & 2021-02-12 & 1487681 & 1488626 & (1174186.41-1877494.66)\\
\hline
331 & 2021-02-13 & 1490063 & 1491973 & (1175219.44-1885824.67)\\
\hline
332 & 2021-02-14 & 1491807 & 1495474 & (1176106.15-1895538.68)\\
\hline
333 & 2021-02-15 & 1492909 & 1499135 & (1176880.67-1906668.9)\\
\hline
334 & 2021-02-16 & 1494119 & 1502966 & (1177565.81-1919261.41)\\
\hline
335 & 2021-02-17 & 1496439 & 1506973 & (1178177.51-1933370.81)\\
\hline
\end{tabular}
\end{table}

\begin{table}

\caption{\label{tab:unnamed-chunk-9}Short-term predictions of total number of reported cases at the national level under the AR1 model. Estimation period 05/03/2020-08/02/2021}
\centering
\begin{tabular}[t]{l|l|r|r|l}
\hline
  & Date & Total & Prediction & Prediction Interval\\
\hline
329 & 2021-02-11 & 1484900 & 1486316 & (1107317.92-1978328.98)\\
\hline
330 & 2021-02-12 & 1487681 & 1489176 & (1108536.68-1984144.68)\\
\hline
331 & 2021-02-13 & 1490063 & 1492104 & (1109578.88-1990818.58)\\
\hline
332 & 2021-02-14 & 1491807 & 1495100 & (1110489.52-1998314.17)\\
\hline
333 & 2021-02-15 & 1492909 & 1498166 & (1111296.59-2006610.93)\\
\hline
334 & 2021-02-16 & 1494119 & 1501301 & (1112019.71-2015697.68)\\
\hline
335 & 2021-02-17 & 1496439 & 1504508 & (1112673.2-2025567.73)\\
\hline
\end{tabular}
\end{table}

\hypertarget{references}{%
\section*{References}\label{references}}
\addcontentsline{toc}{section}{References}

\hypertarget{refs}{}
\leavevmode\hypertarget{ref-Feynman1963118}{}%
1. Feynman RP, Vernon Jr. FL. The theory of a general quantum system
interacting with a linear dissipative system. Annals of Physics.
1963;24: 118--173.
doi:\href{https://doi.org/10.1016/0003-4916(63)90068-X}{10.1016/0003-4916(63)90068-X}

\leavevmode\hypertarget{ref-Dirac1953888}{}%
2. Dirac PA. The lorentz transformation and absolute time. Physica.
1953;19: 888--896.
doi:\href{https://doi.org/10.1016/S0031-8914(53)80099-6}{10.1016/S0031-8914(53)80099-6}

\nolinenumbers


\end{document}

